\subsection{Supported Operating Systems}

The following Linux-based Operating Systems are supported for running a authority node: 

\begin{itemize}
    \item Ubuntu Server 18.04 LTS
    \item Debian 9.8
    \item CentOS 7
\end{itemize}

Affiliates can apply for a certain operating system, but in order to keep a heterogeneous infrastructure EWF NetOps can instruct the Affiliate to use a specific operating system from the above list. 
The operating system must be installed according to the settings described in this document to qualify the host for becoming part of the authority network.


\subsection{Security Requirements}

Running an authority nodes requires raised awareness of host and node security as authorities are a main attack surface to disturb operation of the block chain.
The following security rules apply:

\begin{itemize}
    \item No services are permitted to run on the same host that are not part of the authority node package
    \item All incoming connections on all ports except SSH (22/tcp) and the P2P (30303/tcp+udp) port have to be firewalled on the host with DROP rules. To guarantee proper network etiquette, incoming ICMP has to be accepted.
    \item SSH access is only allowed for non-root users
    \item SSH access is only allowed through RSA keys
    \item Parity RPC endpoints (HTTP, WebSocket) have to be disabled for external use (only docker stack internal)
    \item System updates have to applied regularly and in a timely manner
    \item Regular run of rootkit detectors
\end{itemize}

Most of these rules will be provided in the following set up guides.