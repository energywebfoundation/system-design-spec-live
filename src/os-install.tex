\section{Operating system installation}

The following section provide a comprehensive guide for installation of one the supported operating systems.
All further deployment procedures are based on the installation results.

\subsection{Ubuntu Server 18.04 LTS}

\subsubsection{On-Premise}

Procedure based on version 18.04.2.

\begin{itemize}
    \item Download the ISO from https://www.ubuntu.com/download/server.
    \item Boot the ISO
    \item Select \textbf{English} as language
    \item Choose a convenient keyboard layout
    \item Choose \textbf{Install Ubuntu}
    \item Let the network auto-configure -or- configure manually if needed. The system needs an internet connection.
    \item Select no proxy and keep the mirror address.
    \item Select \textbf{Use an entire disk} and confirm
    \item Choose user name and host name in next screen. Choose a strong password.
    \item Select \textbf{Install OpenSSH Server} but don't import keys   
    \item Don't select any snaps and continue
    \item Finish installation and let it boot to the prompt
    \item Login as the created user and run a full system update using \texttt{sudo apt update \&\& sudo apt dist-upgrade -y}
\end{itemize}

\subsubsection{AWS}

The Ubuntu AMI Id's for all regions in AWS can be found in Table \ref{ubuntuami}

Also Ubuntu AMI's are listed at https://cloud-images.ubuntu.com/locator/ec2/. Search for \textbf{ebs 18.04 amd64} to get the correct version.

\begin{table}[]
\centering
\begin{tabular}{ll}
Region         & AMI ID                \\
ap-northeast-1 & ami-0eb48a19a8d81e20b \\
ap-south-1     & ami-007d5db58754fa284 \\
ap-southeast-1 & ami-0dad20bd1b9c8c004 \\
ca-central-1   & ami-01b60a3259250381b \\
eu-central-1   & ami-090f10efc254eaf55 \\
eu-north-1     & ami-5e9c1520          \\
eu-west-1      & ami-08d658f84a6d84a80 \\
sa-east-1      & ami-09f4cd7c0b533b081 \\
us-east-1      & ami-0a313d6098716f372 \\
us-west-1      & ami-06397100adf427136 \\
cn-northwest-1 & ami-09b1225e9a1d84e4c \\
cn-north-1     & ami-09dd6088c3e46151c \\
us-gov-west-1  & ami-66bdd307          \\
us-gov-east-1  & ami-7bd2340a          \\
ap-northeast-2 & ami-078e96948945fc2c9 \\
ap-southeast-2 & ami-0b76c3b150c6b1423 \\
eu-west-2      & ami-07dc734dc14746eab \\
us-east-2      & ami-0c55b159cbfafe1f0 \\
us-west-2      & ami-005bdb005fb00e791 \\
ap-northeast-3 & ami-0babd61cf592f1c03 \\
eu-west-3      & ami-03bca18cb3dc173c9
\end{tabular}
\caption{Ubuntu 18.04 LTS AWS AMI Id's}
\label{ubuntuami}
\end{table}

\subsubsection{Azure}

The URN for the image is \texttt{Canonical:UbuntuServer:18.04-LTS:latest}

\subsection{Debian 9.8}

\subsubsection{On-Premise}

\begin{itemize}
    \item Download the NetInst ISO from https://www.debian.org/distrib/netinst
    \item Boot the ISO
    \item Select \textbf{Install} from the boot screen
    \item Select \textbf{English} as language
    \item Select Location based on actual location of the host
    \item Chose a convenient keyboard layout
    \item Let the network auto-configure -or- configure manually if needed. The system needs an internet connection.
    \item Name your host. Change it from \textit{debian} to something else
    \item Choose a strong root password
    \item create the user account and choose a strong password
    \item select the proper timezone
    \item For the partitions use \textbf{Guided - use entire disk}
    \item Select \textbf{All files in one partition}
    \item Finish partitioning and write changes to disk
    \item Select \textbf{No} when ask to scan more disks
    \item Choose a mirror close to the host
    \item Opt-out of the package survey
    \item on the \textbf{Software Selection} select only \textbf{SSH Server} and \textbf{standard system utilities}
    \item Install the grub bootloader to MBR and use the primary disk for that
    \item Finish installation and let it boot to the prompt
    \item Login as \texttt{root} and run a full system update using \texttt{apt update \&\& apt dist-upgrade -y}
    \item Reboot
\end{itemize}

\subsubsection{AWS}

The AMI Id's for all regions in AWS can be found in Table \ref{debami}

\begin{table}[]
\centering
\begin{tabular}{ll}
Region         & AMI ID                \\
eu-north-1     & ami-043a919b6dc7c51cc \\
ap-south-1     & ami-0b6490868957ce747 \\
eu-west-3      & ami-0cb185e7696ffe300 \\
eu-west-2      & ami-0ef10a4062f24d89d \\
eu-west-1      & ami-035c67e6a9ef8f024 \\
ap-northeast-2 & ami-0fa1392d5d545f9e8 \\
ap-northeast-1 & ami-0c4290d7ce45d7bbe \\
sa-east-1      & ami-0bc0ce4ab8b82305c \\
ca-central-1   & ami-0857efbad274a1a89 \\
ap-southeast-1 & ami-04c9740a9ed018dba \\
ap-southeast-2 & ami-0b91189c4f9f5cd9e \\
eu-central-1   & ami-05449f21272b4ee56 \\
us-east-1      & ami-0f9e7e8867f55fd8e \\
us-east-2      & ami-00c5940f2b52c5d98 \\
us-west-1      & ami-0afda78f1d0272d99 \\
us-west-2      & ami-01d07e14f082b3ba1
\end{tabular}
\caption{Debian 9.8 AWS AMI's}
\label{debami}
\end{table}

You can retrieve this list also from https://wiki.debian.org/Cloud/AmazonEC2Image/Stretch

\subsubsection{Azure}

The URN for the image is \texttt{credativ:Debian:9:latest}

\subsection{CentOS 7}

\subsubsection{On-Premise}

\begin{itemize}
    \item Download the minimal ISO from https://www.centos.org/download/
    \item Boot the ISO
    \item Confirm the automatic boot option \textbf{Test this media \& install CentOS 7}
    \item Choose \textbf{English} as language
    \item On the installation summary choose "Installation destination" and confirm "automatic partinioning"
    \item Back on the installation summary screen click on "Network \& Hostname"
    \item change the hostname
    \item enable the network interface and make sure it is configured properly
    \item Click \textbf{Done} to get back to the summary and click \textbf{Begin Installation}
    \item During installation set a root password 
    \item Finish installation and let it boot to the prompt
    \item Login as \texttt{root} and run a system update with \texttt{yum update}
\end{itemize}

\subsubsection{AWS}

The AMI Id's for all regions in AWS can be found in Table \ref{centami}

\begin{table}[]
\centering
\begin{tabular}{ll}
Region         & AMI ID       \\
ap-northeast-1 & ami-25bd2743 \\
ap-northeast-2 & ami-7248e81c \\
ap-south-1     & ami-5d99ce32 \\
ap-southeast-1 & ami-d2fa88ae \\
ap-southeast-2 & ami-b6bb47d4 \\
ca-central-1   & ami-dcad28b8 \\
eu-central-1   & ami-337be65c \\
eu-west-1      & ami-6e28b517 \\
eu-west-2      & ami-ee6a718a \\
eu-west-3      & ami-bfff49c2 \\
sa-east-1      & ami-f9adef95 \\
us-east-1      & ami-4bf3d731 \\
us-east-2      & ami-e1496384 \\
us-west-1      & ami-65e0e305 \\
us-west-2      & ami-a042f4d8 \\
\end{tabular}
\caption{CentOS 7 AWS AMI's}
\label{centami}
\end{table}

You can retrieve the list also from \\
https://wiki.centos.org/Cloud/AWS\#head-78d1e3a4e6ba5c5a3847750d88266916ffe69648

\subsubsection{Azure}

The URN for the image is \texttt{OpenLogic:CentOS:7.5:latest}