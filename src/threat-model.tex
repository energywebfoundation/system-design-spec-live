\section{Threat model}

This section describes potential attack vectors to the system and how they are either mitigated by the system design, automatic intervention or human intervention.

\subsection{Telemetry}

As telemetry needs to flow from the validators to a virtual single ingress point on the SaaS provider. It is not protected by decentralization.
This telemetry helps to detect abnormalities in the operation of the validators caused by either system malfunction or deliberate attacks.

\begin{description}
    \item[Sending tampered data] 
        If an attacker manages to disturb node operation, he might try to disguise this by sending normal looking telemetry to the telemetry ingress on behalf of the attacked node. NetOps would detect the faulty node with increased time delay as telemetry data would contradict actual node behavior. \\
        \textbf{Acknoledged:} The current SaaS system won't protect against this attack. 
    
    \item[Denial-of-Service against the telemetry ingress] 
        An attacker might try to bring down the telemetry ingress completely to "blind" the EWF NetOps team about the status of the validator network. \\
        \textbf{Acknoledged:} The SaaS provider should be able to handle an attack against his infrastructure. But this is not proofed.

    \item[Phishing for telemetry] 
        An attacker might try to receive telemetry from a node by re-routing telemetry traffic to an attacker system that mimics the ingress (DNS spoofing, MITM). The attacker can gain knowledge about for example the systems load patterns or how a system might respond to an attack. \\
        \textbf{Acknoledged:} The SaaS provider should be able to handle an attack against his infrastructure.
        
    \item[DNS-Spoof/MitM against GUI] 
        An attacker might manage to DNS Spoof/MitM the connection between the browser of a member from the NetOps team and the telemetry backend. This way the attacker could present a faked system state of the validator network to NetOps. \\
        \textbf{Acknoledged:} The SaaS provider is not able to provide signed data. Security relies on an untampered HTTPS connection.

    \item[Unauthorized access to telemetry data]
        Telemetry data should only be available to the EWF NetOps Team. An attacker might try to directly call the telemetry backend to receive the data. \\
        \textbf{Acknoledged:} The SaaS provider has to authenticate the users correctly.

\end{description}

\subsection{Node Control}

The Node control component is used to carry out updates to the validator nodes. These updates need to be legitimized by the EWF NetOps and GovOps teams.
The component has the potential to disable a node due to a faulty or malicious update .

\begin{description}
    \item[Send fake update event] 
        An attacker might try to fool the NodeControl into carrying out an update script that is not approved. \\
        \textbf{Prevention:} NodeControl will talk to its local blockchain client most of the time via IPC. 
        If it has to talk to outside nodes because the local client is down, it'll use incubed to verify responses.

    \item[Compromised Payload]
        An attacker might MitM traffic to any server NodeControl would download a payload from (mainly used for the chain spec file update process). \\
        \textbf{Containment:} Each update will have the SHA256 hash of its agreed payload on chain. The payload downloaded by NodeControl is verified against this hash. If the hashes don't match NodeControl will not carry out the command and report a faulty update to the NodeControl Contract.

\end{description}


\subsection{Other Attack Vectors}

This section consolidates attack vectors that are either not specific to a single component or targeted against miscellaneous components.

\begin{description}

    \item[Attack time servers] 
        The Aura PoA consensus algorithm highly depends on accurate system clocks. An attacker could try to manipulate the system clock by intercepting requests made by the operating system via NTP (Network Time Protocol). If the attacker shifts the system time by +block time or -block time the validator will start signing during the wrong slot. Other validators then won't accept that block because it is out-of-turn and vice-versa the manipulated node would mark all incoming blocks as invalid as from its point of view all other validators appear to be syncing out-of-turn. \\
        \textbf{Mitigation:} Some validator nodes should use high-precision hardware clocks such as GPS-based ones or DCF-77 clocks instead of NTP for their time source. This way not the whole network is affected. Normal operation could be restored by removing all non-precision-clock nodes from the network. This could mean reduced chain performance.

\end{description}

